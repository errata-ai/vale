\documentclass{article}

\usepackage{fancyhdr}
\usepackage{amssymb}
\usepackage{extramarks}
\usepackage{amsmath}
\usepackage{amsthm}
\usepackage{amsfonts}
\usepackage{tikz}
\usepackage{enumerate}
\usepackage{paralist}
\usepackage{hyperref}
\usepackage{listings}
\usepackage[plain]{algorithm}
\usepackage{algpseudocode}
%\usepackage[]{algorithm2e}

\usetikzlibrary{automata,positioning}

%
% Basic Document Settings
%

\topmargin=-0.45in
\evensidemargin=0in
\oddsidemargin=0in
\textwidth=6.5in
\textheight=9.0in
\headsep=0.25in

\linespread{1.1}

\pagestyle{fancy}
\lhead{\hmwkAuthorName}
% EDITED: \chead{\hmwkClass\ (\hmwkClassInstructor\ \hmwkClassTime): \hmwkTitle}
\chead{\hmwkClass: \hmwkTitle}
\rhead{\firstxmark}
\lfoot{\lastxmark}
\cfoot{\thepage}

\renewcommand\headrulewidth{0.4pt}
\renewcommand\footrulewidth{0.4pt}

\renewcommand*{\theenumi}{\alph{enumi}.}
\renewcommand*{\labelenumi}{\theenumi~}

\setlength\parindent{0pt}

%
% Create Problem Sections
%

\newcommand{\enterProblemHeader}[1]{
    \nobreak\extramarks{}{Exercise {#1} continued on next page\ldots}\nobreak{}
    \nobreak\extramarks{Exercise {#1} (continued)}{Exercise {#1} continued on next page\ldots}\nobreak{}
}

\newcommand{\exitProblemHeader}[1]{
    \nobreak\extramarks{Exercise {#1} (continued)}{Exercise {#1} continued on next page\ldots}\nobreak{}
    \nobreak\extramarks{Exercise \arabic{#1}}{}\nobreak{}
}

%
% Homework Problem Environment
%
% This environment takes an optional argument. When given, it will adjust the
% problem counter. This is useful for when the problems given for your
% assignment aren't sequential. See the last 3 problems of this template for an
% example.
%
\newenvironment{homeworkProblem}[1][-1]{
    \nobreak\extramarks{Page \thepage}{}\nobreak{}
    \section{Exercise {#1}}
    %\enterProblemHeader{#1}
}{
    %\exitProblemHeader{#1}
}

%
% Homework Details
%   - Title
%   - Due date
%   - Class
%   - Section/Time
%   - Instructor
%   - Author
%

\newcommand{\hmwkTitle}{Set Theory Drill Assignment}
\newcommand{\hmwkDueDate}{\today}
\newcommand{\hmwkClass}{CS 251}
\newcommand{\hmwkClassTime}{}
\newcommand{\hmwkClassInstructor}{}
\newcommand{\hmwkAuthorName}{Joseph Kato}

\algnewcommand\True{\textbf{true}\space}
\algnewcommand\False{\textbf{false}\space}

%
% Title Page
%

\title{
    \vspace{2in}
    \textmd{\textbf{\hmwkClass:\ \hmwkTitle}}\\
    \normalsize\vspace{0.1in}{\hmwkDueDate}\\
    \vspace{0.1in}\large{\textit{\hmwkClassInstructor\ \hmwkClassTime}}
    \vspace{3in}
}

\author{\textbf{\hmwkAuthorName}}
\date{}

\renewcommand{\part}[1]{\textbf{\large Part \Alph{partCounter}}\stepcounter{partCounter}\\}

%
% Various Helper Commands
%

% Useful for algorithms
\newcommand{\alg}[1]{\textsc{\bfseries \footnotesize #1}}

% For derivatives
\newcommand{\deriv}[1]{\frac{\mathrm{d}}{\mathrm{d}x} (#1)}

% For partial derivatives
\newcommand{\pderiv}[2]{\frac{\partial}{\partial #1} (#2)}

% Integral dx
\newcommand{\dx}{\mathrm{d}x}

% Alias for the Solution section header
\newcommand{\solution}{\textbf{\large Solution}}

% Probability commands: Expectation, Variance, Covariance, Bias
\newcommand{\E}{\mathrm{E}}
\newcommand{\Var}{\mathrm{Var}}
\newcommand{\Cov}{\mathrm{Cov}}
\newcommand{\Bias}{\mathrm{Bias}}

\begin{document}

%\maketitle
%\pagebreak

\begin{homeworkProblem}[6.1.1]
    In each of (a)-(f), answer the following questions: Is A $\subseteq$ B? Is
    B $\subseteq$ A? Is either A or B a proper subset of the other? \textbf{XXX}

    \begin{enumerate}[a.]
        \item A = \{2, \{2\}, $(\sqrt{2})^2$\}, B = \{2, \{2\}, \{\{2\}\}\}
        \item A = \{3, $\sqrt{5^2 - 4^2}$, 24 \(mod\) 7\}, B = \{8 \(mod\) 5\}
        \item A = \{\{1, 2\}, \{2, 3\}\}, B = \{1, 2, 3\}
        \item A = \{a, b, c\}, B = \{\{a\}, \{b\}, \{c\}\}
        \item A = \{$\sqrt{16}$, \{4\}\}, B = \{4\}
        \item A = \{$x \in \mathbb{R} ~\mid~ cos(x) \in \mathbb{Z}$\},
              B = \{$x \in \mathbb{R} ~\mid~ sin(x) \in \mathbb{Z}$\}
    \end{enumerate}

    \textbf{Solution}

    \begin{enumerate}[a.]
        \item A $\subseteq$ B because every element in A is in B
        (note: $(\sqrt{2})^2$ = 2), but B $\nsubseteq$ A because \{\{2\}\}
        $\in$ B and \{\{2\}\} $\notin$ A. Because A $\subseteq$ B and
        B $\nsubseteq$ A, A is a proper subset of B.
        \item A = \{3, $\sqrt{5^2 - 4^2}$, 24 \(mod\) 7\} = \{3, 3, 3\} = \{3\}
        and B = \{8 \(mod\) 5\} = \{3\}. So, A $\subseteq$ B and B $\subseteq$
        A, but neither is a proper subset of the other.
        \item A $\nsubseteq$ B because \{1, 2\} $\in$ A and \{1, 2\} $\notin$ B,
        and B $\nsubseteq$ A because 1 $\in$ B and 1 $\notin$ A. Neither is a
        proper subset of the other.
        \item A $\nsubseteq$ B because a $\in$ A and a $\notin$ B,
        and B $\nsubseteq$ A because \{a\} $\in$ B and \{a\} $\notin$ A. Neither
        is a proper subset of the other.
        \item A = \{$\sqrt{16}$, \{4\}\} = \{4, \{4\}\}. So, A $\nsubseteq$ B
        because \{4\} $\in$ A and \{4\} $\notin$ B, but B $\subseteq$ A. B is
        also a proper subset of A FIXME because A $\nsubseteq$ B.
        \item A $\subseteq$ B and B $\subseteq$ A, but neither is a proper
        subset of the other.
    \end{enumerate}

\end{homeworkProblem}

\begin{homeworkProblem}[6.1.18]
    \begin{inparaenum}
        \item\label{q1a} Is the number 0 in $\emptyset$? ~~~
        \item\label{q1b} Is $\emptyset$ = \{$\emptyset$\}? ~~~
        \item\label{q1b} Is $\emptyset \in$ \{$\emptyset$\}? ~~~
        \item\label{q1b} Is $\emptyset \in \emptyset$? ~~~
    \end{inparaenum}\newline

    \textbf{Solution}

    \begin{enumerate}[a.]
        \item No, since by defintion $\emptyset$ has no elements, 0 is not
        in $\emptyset$.
        \item No, $\emptyset$ is the empty set and \{$\emptyset$\} is a set
        with a single element.
        \item Yes, $\emptyset \in$ \{$\emptyset$\}.
        \item No, since by defintion $\emptyset$ has no elements, $\emptyset \notin \emptyset$.
    \end{enumerate}
\end{homeworkProblem}

\pagebreak

\begin{homeworkProblem}[6.1.23]
    Let $V_{i}$ = $\left( x \in \mathbb{R} ~\mid~ \frac{-1}{i} \leq x \leq \frac{1}{i} \right)$
    = $\left[ \frac{-1}{i}, \frac{1}{i} \right]$ for all possible integers \(i\). \newline

    \begin{inparaenum}
        \item\label{q1a} $\bigcup \limits_{i=1}^4 V_{i}$ =? ~~~
        \item\label{q1b} $\bigcap \limits_{i=1}^4 V_{i}$ =? ~~~
        \item\label{q1b} Are $V_{1}, V_{2}, V_{3}...$ mutually disjoint? ~~~
        \item\label{q1b} $\bigcup \limits_{i=1}^n V_{i}$ =? ~~~
        \item\label{q1b} $\bigcap \limits_{i=1}^n V_{i}$ =? ~~~
        \item\label{q1b} $\bigcup \limits_{i=1}^\infty V_{i}$ =? ~~~
        ~~\item\label{q1b} $\bigcap \limits_{i=1}^\infty V_{i}$ =? ~~~
    \end{inparaenum}\newline

    \textbf{Solution}\newline

    \begin{inparaenum}
        \item\label{q1a} $\bigcup \limits_{i=1}^4 V_{i}$ = $\left[-1, 1\right]$~~~
        \item\label{q1a} $\bigcap \limits_{i=1}^4 V_{i}$ = $\left[\frac{-1}{4}, \frac{1}{4}\right]$~~~
        \item\label{q1a} No, $V_{1}, V_{2}, V_{3}...$ are not mutually disjoint. \newline
        \item\label{q1a} $\bigcup \limits_{i=1}^n V_{i}$ = $\left[-n, n\right]$.~
        \item\label{q1a} $\bigcap \limits_{i=1}^n V_{i}$ = $\left[\frac{-1}{n}, \frac{1}{n}\right]$.~
        ~\item\label{q1a} $\bigcup \limits_{i=1}^\infty V_{i}$ = $\left[-1, 1\right]$. ~~
        \item\label{q1a} $\bigcap \limits_{i=1}^\infty V_{i}$ = \{0\}. ~~~
    \end{inparaenum}\newline
\end{homeworkProblem}

\begin{homeworkProblem}[6.1.38]
    Write an algorithm to determine whether a given element \(x\) belongs to a
    given set, which is TODO represented as an array a[1], a[2], ..., a[n].\newline

    \textbf{Solution}\newline

    \textbf{\textcolor{blue}{Input:}} x, a[1], a[2], ..., a[n] [\(a~one-dimensional~array\)]\newline
    \\
    \textbf{\textcolor{blue}{Algorithm Body: }}\newline

    \begin{algorithmic}
    \State $i := 1, found :=$ \False
    \While {$\left(i \leq n~and~found = \False\right)$}
        \If {$a[i] = x$}
            \State $found :=$ \True
        \EndIf
        \State $i := i + 1$
    \EndWhile\newline
    \end{algorithmic}
    \textbf{\textcolor{blue}{Output: found [a boolean]}}
\end{homeworkProblem}

\begin{homeworkProblem}[6.2.39]
    For all integers $n \geq 1$, if $A_{1}, A_{2}, A_{3}$, ... and B are any sets,
    then \newline

    $\bigcap \limits_{i=1}^n \left(A_{i} - B \right)$ = $\left(\bigcap \limits_{i=1}^n A_{i} \right) - B$
    \newline

    \textbf{Solution}\newline


\end{homeworkProblem}

\end{document}
